\section{Useful Information}

A full stop (.) is used at the end of a sentence.\\

A comma (,) is used:
\begin{itemize}
\item like a brief pause in speech to make the sentence easier to read.
to separate items in a list, except for the last two items where we use and.
\item after many linking words that come at the beginning of a sentence,\\
e.g. However, Also, Moreover, In addition, For example

\item before please if it’s used at the end of a sentence,\\
e.g. Can you provide me with some examples of this use, please?

However, at the beginning of a sentence please is usually not separated by a comma,\\
e.g. Please provide me with examples.

\item After a salutation and a closing line. However, in modern emails commas may be dropped in these cases,\\
e.g. Dear Mary (); Best regards ( ).
\end{itemize}

A colon (:) is used to introduce items in a list, e.g. action items.\\

A semi-colon (;) is used to separate long items in a list, particularly if there are commas inside some items.\\

Capital letters (upper case) are used:
\begin{itemize}
\item to begin a sentence;
\item for names of people (Kate Smith), places (Minsk), events (IT Week) and organizations (EPAM);
\item for job titles (Business Analyst); 
\item for nationalities (Belarusian);
\item for calendar information like days (Friday), months (September), etc.
\end{itemize}